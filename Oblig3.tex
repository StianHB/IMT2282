\documentclass[11pt]{article}
\usepackage[utf8]{inputenc}
\usepackage{hyperref}
\usepackage[norsk]{babel}
\DeclareUnicodeCharacter{00A0}{ }
\setcounter{section}{10}

\title{IMT2282 - Operativsystemer \\
	Oblig 2}
\author{Stian Hoel Bergseth \\
	130378}

\date{\today}


\begin{document}
\maketitle

\section{Chapter 11}
\subsection{Theory Questions}
\setcounter{subsection}{6}
\subsubsection{1. Hva er forskjellen på memory mapped og isolated I/O? Angi fordeler og ulemper med disse to prinsippene.}

I/O enheter, som f.eks disker, har egne registere for å ta imot og sende fra seg data. Hvis man skal snakke direkte til disse, isolated I/O , krever det at man har egne instruksjoner for å sende/hente data fra disse registrene. Hvis man heller bruker memory mapped I/O så kan man bruke de samme instruksjonene for å snakke med hurtigminnet og I/O enhetene. Dette ville krevd egne funksjoner for å komunisere med I/O enheter og ville skapt ekstra overhead sammenlignet med å bare kunne aksessere enhetene som variabler i hurtigminnet. En annen fordel er at man unngår behovet for egne sikkerhetsmekanismer for I/O enheter. Med memory mapped I/O så holder det at operativsystemet ikke tilegner prosesser minneadressene som er reservert til I/O. En av ulempene ved memory mapped I/O er at både hurtigminne og I/O enhetene må kunne få med seg alle forespørrsler CPUen gjør på adresse bus'en. Her har vi et problem ettersom hurtigminnet er raskt, mens I/O enheter er vesentlig treigere. Dette har man kommet rundt ved å ha en egen dedikert bus mellom CPU og RAM, mens I/O enhetene kan enten lytte på denne bus'en( men da vil ikke I/O enhetene klare å henge med ) eller så kan memory controlleren ha et predefinert sett med adresser som den flytter ut på en I/O bus hvis CPUen ønsker å snakke til en I/O enhet.

Isolated I/O

\subsubsection{2. På en harddisk, hvor mange bytes finnes som regel i en sektor? Hva er en sylinder? Hva er typisk gjennomsnittlig aksesstid for en disk i dag? Hva er overføringsraten (ca, i MB/s) mellom diskplate og buffer?}
\subsubsection{3. Hva oppnår vi med å koble diskene som RAID disker? Hvordan er diskene organisert på RAID-level 1. Forklar hvordan diskene er organisert på RAID-level 5.}
\subsubsection{4. Hvilke fire kriterier definerer et presist interrupt?}
\subsubsection{5. Forklar forskjellen mellom HDD og SSD når det gjelder lesing, skriving/overskriving og sletting av filer. Hva er poenget med TRIM kommandoen?}
\subsubsection{6. Hva betyr det at et operativsystem er tilpasset SSD disker (slik som f.eks. Windows 7 er).}
\end{document}
